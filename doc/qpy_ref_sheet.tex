\documentclass[a4paper,11pt]{article}

\usepackage[top=1.0cm,left=1.0cm,right=1.0cm,bottom=1.0cm]{geometry}
\linespread{1.25}

\usepackage[utf8]{inputenc}
\usepackage[english]{babel}

\usepackage[pdftex]{graphicx}
\usepackage{pdflscape}

\usepackage{multirow}
\usepackage{color}
\usepackage{colortbl}

\usepackage{rotating}


\definecolor{RC1}{gray}{0.92}
\definecolor{RC2}{rgb}{1.0,0.5,0.5}
\definecolor{RC3}{gray}{1.0}

\newcommand{\qpy}{{\texttt{q}\raisebox{-0.45ex}{\textit{py}}}}

\usepackage[most]{tcolorbox}
\tcbset{
    frame code={}
    center title,
    left=0pt,
    right=0pt,
    top=0pt,
    bottom=0pt,
    colback=gray!90,
    colframe=white,
    width=\dimexpr\textwidth\relax,
    enlarge left by=0mm,
    boxsep=5pt,
    arc=0pt,outer arc=0pt,
    }

\usepackage{tikz}
\usetikzlibrary{calc}
\usetikzlibrary{positioning}
\usetikzlibrary{arrows.meta}

\begin{document}
\pagenumbering{gobble}
\begin{landscape}

  \colorbox{RC2}{\textbf{\Huge Cheat Sheet}} \: {\Huge \qpy{}} \: The \textbf{Q}ueue system of \textbf{P}radipta and \textbf{Y}uri made in \textbf{Py}thon - version 0.0, 2018\vspace{0.0cm}

  \begin{center}
    \colorbox{RC2}{Basic usage:} {\texttt{qpy <command> [options]}} \colorbox{RC1}{Use \texttt{TAB} for completion and \texttt{?} \texttt{TAB TAB} to display a help}
  \end{center}\vspace{0.0cm}
  
  % ==========
  \begin{minipage}{1.0\textheight}
    \begin{tabular}{lllll}
      \hline
      \rowcolor{RC2}
      Command            & Options & \multicolumn{3}{l}{Explanation} \\
      \hline
      \texttt{restart}   &
                                   & \multicolumn{3}{l}{(re)starts the background environment for \qpy{}} \\
      \rowcolor{RC1}
      \texttt{sub}       & \texttt{[-n <\# cores>]} \texttt{[-m <memory, in GB>]} \texttt{<command>}
                                   & \multicolumn{3}{l}{submits the command \texttt{<command>}, optionally giving the number of cores and memory}\\
      \texttt{status}    &
                                   & \multicolumn{3}{l}{shows the status of nodes and users}\\
      \cellcolor{RC1}\texttt{check} & \cellcolor{RC1}\texttt{[<status>] [<dir>] [<job IDs>]}
                                   & \cellcolor{RC1}lists ... & \multirow{3}{0.3cm}{$\left\}\begin{array}{l} \\ \\ \\ \end{array}\right.$}
                                   & \cellcolor{RC1}... all the jobs with the given status(es), or that was submitted from\\
      \texttt{clean}     & \texttt{[<status>] [<dir>] [<job IDs>]}
                                   & cleans ... &
                                   & \cellcolor{RC1}given directory(ies), or has ID given in \texttt{<job IDs>}. If no option\\
      \cellcolor{RC1}\texttt{kill} & \cellcolor{RC1}\texttt{[<status>] [<dir>] [<job IDs>]}
                                   & \cellcolor{RC1}kills ... &
                                   & \cellcolor{RC1}is given, \texttt{clean} and \texttt{kill} have no effect, whereas \texttt{check} lists all jobs\\
      \texttt{config}    & \texttt{[checkFMT <pattern>]}
                                   & \multicolumn{3}{l}{shows the pattern used for check, or sets it to \texttt{<pattern>}} \\
                         & \texttt{[colour <false|true>]}
                                   & \multicolumn{3}{l}{turns the output of the \texttt{check} command coloured (\texttt{true}) or disable the colours (\texttt{false})} \\
                         & \texttt{[colourScheme <colour 1> ... <colour 5>]}
                                   & \multicolumn{3}{l}{shows the colours used in the output of the \texttt{check} command, or sets them} \\
      \rowcolor{RC1}
      \texttt{ctrlQueue} & \texttt{pause}
                                   & \multicolumn{3}{l}{pauses the submission of jobs} \\
      \rowcolor{RC1}
                         & \texttt{continue}
                                   & \multicolumn{3}{l}{continues the submission of jobs} \\
      \rowcolor{RC1}
                         & \texttt{jump <job IDs> <target>}
                                   & \multicolumn{3}{l}{moves jobs with IDs in \texttt{<job IDs>} to \texttt{<target>}, that can be a ID, \texttt{begin}, or \texttt{end}} \\
      \texttt{tutorial}  & \texttt{[<keyword>]}
                                   & \multicolumn{3}{l}{shows the tutorial, at \texttt{<keyword>}} \\
      \rowcolor{RC1}
      \texttt{finish}    &
                                   & \multicolumn{3}{l}{finishes the background environment for \qpy{}} \\
      \hline
    \end{tabular}
  \end{minipage}\vspace{0.2cm}

  % ==========
  \begin{minipage}{0.55\textwidth}
    \begin{tabular}{ll}
      \hline
      \rowcolor{RC2}
      \multicolumn{2}{l}{Possible job statuses}\\
      \hline
      \verb+queue+ & Job in the queue, not running yet\\
      \rowcolor{RC1}
      \verb+running+ & Job being executed\\
      \verb+done+ & Job has finished\\
      \rowcolor{RC1}
      \verb+undone+ & Job was removed from the queue before running\\
      \verb+kill+ & Job was killed when running\\
      \hline
    \end{tabular}\vspace{0.2cm}
    \begin{center}
      \begin{tabular}{ll}
        \hline
        \rowcolor{RC2}
        \multicolumn{2}{l}{Environment variables}\\
        \hline
        \verb+QPY_JOB_ID+ & job ID\\
        \rowcolor{RC1}
        \verb+QPY_NODE+ & node where job is running\\
        \verb+QPY_N_CORES+ & number of requested cores\\
        \rowcolor{RC1}
        \verb+QPY_MEM+ & requested memory\\
        \hline
      \end{tabular}
    \end{center}
  \end{minipage}\quad
  % ==========
  \begin{minipage}{0.4\textwidth}
    \begin{tabular}{ll}
      \hline
      \rowcolor{RC2}
      \multicolumn{2}{l}{Possible modifiers to be used in \texttt{<pattern>} for \texttt{config checkFMT}}\\
      \hline
      \verb+%j+ & job ID\\
      \rowcolor{RC1}
      \verb+%s+ & job status\\
      \verb+%c+ & command used to submit the job\\
      \rowcolor{RC1}
      \verb+%d+ & working directory of the job\\
      \verb+%n+ & node allocated for the job\\
      \rowcolor{RC1}
      \verb+%N+ & number of cores of the job\\
      \verb+%Q+ & time when the job was submitted\\
      \rowcolor{RC1}
      \verb+%S+ & time when the job started to run\\
      \verb+%E+ & time when the job has finished\\
      \rowcolor{RC1}
      \verb+%R+ & the time in queue, or the running time, or the total running time\\
      \rowcolor{RC1}
                & (depending on the status of the job) \\
      \hline
    \end{tabular}
  \end{minipage}
  % ==========
  \colorbox{RC3}{
      \begin{minipage}{0.41\textwidth}
        \begin{tabular}{ll}
          \hline
          \rowcolor{RC2}
          Syntax for sets of job IDs\\
          \hline
          \texttt{<initial ID>-<final ID>,<ID1> <ID2>}\\
          Example: \texttt{10-14,4,30 20}, means\\
          10, 11, 12, 13, 14, 4, 30, and 20\\
          \hline
        \end{tabular}
      \end{minipage}
  }
  
\end{landscape}
\end{document}

  
%%% Local Variables:
%%% ispell-local-dictionary: "british"
%%% mode: latex
%%% TeX-master: t
%%% End:
